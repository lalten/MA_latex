%LaTeX
\typeout{$Id$}
%
%%%%%%%%%%%%%%%%%%%%%%%%%%%%%%%%%%%%%%%%%%%%%%%%%%%%%%%%%%%%%%%%%%%%%%
%&delatex
\documentclass[%
	12pt,		% 12pt font size
	headinclude,	% 
	%openany,	% Kapitel können auch auf der linken Seite anfangen.
	a4paper,	% Papierformat
	pointlessnumbers, % entgegen Duden Nr. 21, S. 21 R4 als CI festgelegt
	changebar,      % Änderungsbalken mit \begin/\end{changebar}
	liststotoc,	% Was alles in der Table of Contents stehen soll
	bibtotoc,
	idxtotoc,
	%draft,		% mark overfull hboxes with a thick rule, don't include
	%		% (EPS) files, begin bibliography on a new page
	]{diplomarbeit}
	% noheadsepline

\areaset[20mm]{160mm}{240mm}	% Binderand, Breite und Höhe nutzbarer Bereich

\usepackage{bibgerm}		% Für geralpha.bst. MUSS vor babel stehen
\usepackage[utf8]{inputenc}	% Latin1 input encoding (umlauts)
%\usepackage{pslatex}
% For english, uncomment the following two lines and comment out the
% two lines below that. Don't forget to 'make clean' when switching 
% between languages.
%\usepackage[german,english]{babel} % Multilingual text, order is important !
%\selectlanguage{english}        % Select default language
\usepackage[german,english]{babel} % Mehrsprachiger Text, Reihenfolge wichtig!
				   % Die zuletzt genannte Sprache wird verwendet!
%  \selectlanguage{english}	% Wähle Standardsprache ...
\usepackage{version}		% handle comments and different versions
				% \begin/end{comment},
				% \include/excludeversion{foo}
\usepackage{longtable}		% \begin{longtable} ... see LaTeX Companion
\usepackage{parskip}		% seperate paragraphs by skip (blank line)
%\usepackage{amsfonts,amsmath,amssymb,array} % choose if needed
%\usepackage{ae,aecompl}	% Benutze 'Almost European' Zeichensatz         
%\usepackage{type1cm}	% Computer Modern type 1 fonts
\usepackage{lmodern}	% Latin Modern nehmen, da mehr Glyphen vorhanden 
\usepackage[T1]{fontenc}	% Verwende T1-Kodierung für Zeichensätze        


\usepackage{changebar}
\usepackage{makeidx}\makeindex

% --------------------------------------------------------------------
% Typ der Arbeit
% --------------------------------------------------------------------
% \arbeitsart{Techreport}
% \techreportident{TR-XXXXXYY}
% \arbeitsart{Studienarbeit}
% \arbeitsart{IDP}
% \arbeitsart{Praktikumsbericht}
% \arbeitsart{Bachelor}
% \arbeitsart{Diplomarbeit}
\arbeitsart{Master's Thesis}

% --------------------------------------------------------------------
% Titel, Autor und sonstige Titelblatt--Angaben:
% \betreuer{}, \band{} und \eingereicht{} setzen die
% Style--Option `datitlepage' voraus!
% --------------------------------------------------------------------
\title{Das ist ein etwas längerer Titel\\
        dieses leeren Diplomarbeitsgerüsts}
\author{Martin Musterdiplomand}
\authorsaddr{Arcisstraße 21\\
             80333 München}

\betreuer{Dipl.--Ing. Muster Betreuer}
% --------------------------------------------------------------------
% bei externer Diplomarbeit
% --------------------------------------------------------------------
%\ausgefuehrtam{bei BMW}
%%% oder
%\ausgefuehrtam{%
%  am Institut für Werkzeugmaschinen und Betriebswissenschaften\\
%   Technische Universität München}
%%%_optional: \band{Band1}
%%%_optional: \eingereicht{im September 2004}

% --------------------------------------------------------------------
% eigene Definitionen f"ur die Praeambel
% --------------------------------------------------------------------
% commonly used shorthands:
\newcommand{\qq}[1]{\glqq#1\grqq}	% Vereinfachung
\newcommand{\zb}{z.\,B.\ }              % Vereinfachung
\newcommand{\zB}{z.\,B.\ }              % manche von uns schreiben lieber zB
\newcommand{\dht}{d.\,h.\ }             % Vereinfachung
\newcommand{\ua}{u.\,a.\ }              % Vereinfachung
\newcommand{\bzw}{bzw.\ }               % Vereinfachung
\newcommand{\ggf}{ggf.\ }               % Vereinfachung
\newcommand{\etc}{etc.\ }               % Vereinfachung
\newcommand{\evtl}{evtl.\ }             % Vereinfachung
\newcommand{\bzgl}{bzgl.\ }             % Vereinfachung
\newcommand{\iA}{i.\,A.\ }              % Vereinfachung
\newcommand{\sog}{sog.\ }               % Vereinfachung
\newcommand{\ca}{ca.\,}
\newcommand{\vgl}{vgl.\,}
\newcommand{\usw}{usw.\ }
\newcommand{\va}{v.\,a.\ }
\newcommand{\idR}{i.\,d.\,R.\ }

% commonly used shorthands:
\newcommand{\Cf}{Cf.\ }
\newcommand{\cf}{cf.\ }
\newcommand{\Eg}{E.g.\ }
\newcommand{\eg}{e.g.\ }
\newcommand{\Ie}{I.e.\ }
\newcommand{\ie}{i.e.\ }
\newcommand{\vs}{vs.\ }

% shorthands for units
%\newcommand{\Bar}[1]{\overline{#1}} %conflicts with ams*
\newcommand{\Unit}[1]{\mbox{\,#1}}

\newcommand{\bit}{\Unit{bit}}
\newcommand{\kB}{\Unit{kB}}
\newcommand{\MB}{\Unit{MB}}
\newcommand{\Mb}{\Unit{MBit}}
\newcommand{\GB}{\Unit{GB}}

\newcommand{\Hz}{\Unit{Hz}}
\newcommand{\kHz}{\Unit{kHz}}
\newcommand{\MHz}{\Unit{MHz}}
\newcommand{\GHz}{\Unit{GHz}}

\newcommand{\ms}{\Unit{ms}}
\newcommand{\ns}{\Unit{ns}}
\newcommand{\us}{\Unit{$\mu$s}}

\newcommand{\prs}{\Unit{s$^{-1}$}}
\newcommand{\MBs}{\MB\prs}
\newcommand{\Mbs}{\Mb\,\prs}
\newcommand{\GBs}{\GB\prs}
\newcommand{\Bins}{\Unit{Bildern}~\prs}
\newcommand{\Bis}{\Unit{Bilder}~\prs}

% specific shorthands

\newcommand{\pla}{Plattform A}
\newcommand{\plb}{Plattform B}
\newcommand{\plc}{Plattform C}



% --------------------------------------------------------------------
% PDF specials
% --------------------------------------------------------------------
	\pdfcompresslevel=9
        \RequirePackage[pdftex,		
		      %% für die DRUCK-Ausgabe bitte folgende
		      %% Einstellungen verwenden
		       pdfpagelabels=true,
                       bookmarks=true,
                       pagebackref=true,       % Rücklinks im Lit.-verz.
                       linktocpage=true,
                       plainpages=false,
                       hyperfootnotes=false,   % sonst doppelt definiert
                       breaklinks=true,
                       colorlinks=true,
                       linkcolor=black,
                       urlcolor=black,
                       citecolor=black]{hyperref} 
		%% für die .pdf-Datei-Ausgabe bitte diese
		%% Einstellungen verwenden und obige auskommentieren! 
%		       pdfpagelabels=true,
%		       bookmarks=true,
%		       pagebackref=true,	% Rücklinks im Lit.-verz.
%		       linktocpage=true,
%		       plainpages=false,
%		       hyperfootnotes=false,	% sonst doppelt definiert
%		       breaklinks=true,
%		       colorlinks=true,
%		       linkcolor=blue]{hyperref}
	\makeatletter
	\hypersetup{
		pdftitle={\@title},
		pdfauthor={\@author},
		pdfsubject={Diplomarbeit},
		pdfkeywords={keyword1,keyword2},
	}
	\makeatother

% wenn nur einzelne Kapitel gedruckt werden sollen...
%
%\includeonly{meineergebnisse}

% --------------------------------------------------------------------
%%% Hier beginnt das eigentliche LaTeX--Dokument
% --------------------------------------------------------------------
\begin{document}
% --------------------------------------------------------------------
% Titelseite
% --------------------------------------------------------------------
\renewcommand{\thepage}{t\arabic{page}}
\maketitle

% --------------------------------------------------------------------
% Dank 
% --------------------------------------------------------------------
 \pagestyle{empty}
\cleardoublepage 
\setcounter{page}{3}	% Laut VDI Richtlinien
\renewcommand{\thepage}{\roman{page}}
\begin{danksagung}
  Vielen Dank \dots

  München, im Monat Jahr
\end{danksagung}

% Hier könnte eine Widmung stehen!
\begin{Widmung}
% oder auch nicht. ;)
\end{Widmung}

% --------------------------------------------------------------------
% Inhaltsverzeichnis, Abbildungsverzeichnis, Tabellenverzeichnis
% --------------------------------------------------------------------
\clearpage
\pagestyle{headings}
% Inhaltsverzeichnis soll (nur) in den PDF Bookmarks erscheinen
\pdfbookmark[1]{\contentsname}{toc}
\tableofcontents

\listoffigures
\listoftables
% --------------------------------------------------------------------
% Liste der verwendeten Symbole
% --------------------------------------------------------------------
\phantomsection
\begin{listofsymbols}
\begin{tabbing}
sp	\= symbol-space---- \= \kill \+ \\

RCS	\> Lehrstuhl für Realzeit-Computersysteme \\
\end{tabbing}
\end{listofsymbols}

% --------------------------------------------------------------------
% Abstract (Kurzfassung)
% In Deutsch und Englisch zu verfassen laut Promotionsordnung, zumindest
% bei elektronischer Veröffentlichung.
% --------------------------------------------------------------------
%\pagestyle{empty}
\cleardoublepage 
\begin{abstract}
Die Kurzfassung \dots
\end{abstract}

% Man muss erzwingen, dass die Seite mit Nummer 1 rechts ist, sonst
% kommt latex aus dem Tritt, weil ungerade Seiten immer rechts sein
% müssen!
\newpage{\pagestyle{plain}\cleardoublepage}
\rmfamily	% ab jetzt mit Serifen, weil besser lesbar
% Die Seitennumerierung wird hier mit "1" begonnen
\renewcommand{\thepage}{\arabic{page}}
\setcounter{page}{1}





% --------------------------------------------------------------------
% Hier beginnt der eigentliche Text
% --------------------------------------------------------------------
% Die Identifikation wesentlicher eigener Beiträge muß schon im
% Inhaltsverzeichnis möglich sein.

%\include{standderforschung}
%\include{meinearbeit}
%\include{meineergebnisse}
%\include{fazit}

\chapter{Kapitel}
\label{chap:Kapitel}
\section{Abschnitt}
\label{sec:Abschnitt}
\subsection{Unterabschnitt}
\label{sub:Unterabschnitt}
\paragraph{Paragraph}
\label{par:Paragraph}

Man kann hier auch einfach \index{Text} schreiben. Die pdf--Thumbnails
werden automatisch erzeugt.

Allerdings sind mehrfache Aufrufe nötig, um die Referenzen (Kapitel
\ref{chap:Kapitel}) und den Index (\textit{makeindex diss} ist auch nötig)
richtig hinzukriegen. Das Skript \qq{texaux} kümmert sich darum.

\chapter{Einleitung}

Hier sollte nun der Text der Diplomarbeit folgen.
Stattdessen folgt eine GANZ KURZE Einführung in \LaTeX.

\section{Überblick}

\TeX ~ ist ein einfaches System, mit dem man sehr schnell gute und optisch
schöne Texte erstellen kann. Ersteres ist hauptsächlich die Arbeit
des Autors, während die Optik zum größten Teil von \TeX\ gemacht wird.

Alle \LaTeX -Kommandos beginnen mit einem ``$\backslash$''.  Welche
Kommandos es gibt, kann man z.B. in {\em \LaTeX ~ eine Einf"uhrung}
von Helmut Kopka (\cite{kopkaEinf:92}) oder in \cite{kopkaErw:92} nachlesen.

\section{Drucken}

Damit man drucken kann, müssen die \LaTeX -Quellen übersetzt
werden. Dazu gibt man einfach {\tt make pdf} oder {\tt make ps}
an. Ist das {\tt Makefile} richtig, sollte alles andere nun
automatisch ablaufen. Drucken kann man nun aus {\tt acroread} (für
PDF-Dokumente) oder mit {\tt lpr diplomarbeit.ps}. Um Papier zu
sparen, empfiehlt es sich für Probeausdrucke, zwei Seiten auf eine zu
drucken: {\tt psnup -2 diplomarbeit.ps | lpr}.

\section{Grafiken}

Die Einbindung von Grafiken ist auch nicht schwer. Dabei muß man
unterscheiden, ob man {\tt pdflatex} oder {\tt latex} verwendet. Bei
{\tt latex} ist der Fall klar -- Grafiken sollten hier generell als
EPS (Encapsulated Postscript) vorliegen. {\tt pdflatex} dagegen kann
direkt PDF, JPEG, PNG (portable network grafics) und TIFF
einbinden. Man sollte die Dateiendung in den \LaTeX -Quellen {\bf
  nicht} angeben, dann werden die richtigen Formate automatisch
gesucht. Ist das {\tt Makefile} korrekt, werden eventuell notwendige
Konvertierungen automatisch ausgeführt. In jedem Fall sollte man sich
vorher noch
\url{http://www.rcs.ei.tum.de/intranet-public/informationen/eps_HOWTO.html}
und
\url{http://www.rcs.ei.tum.de/intranet-public/dokumentationen/pdflatex.html}
durchlesen. Zum Zeichnen der Bilder selber nutzt man z.~B. {\tt xfig}.


Hierzu \footnote{test} wurde das Makro-Paket {\tt graphicx.sty} verwendet.

\section{Tabellen}

Werden Tabellen länger als eine Seite muss man z.~B.\ das {\tt
longtable} Paket verwenden. Sonst gehts wie folgt:

\begin{table}[!ht]
 \begin{center}
  \begin{tabular}{|l|r|}
        \hline
        Links & Rechts \\
        \hline
        \hline
        Eintrag 1 & Eintrag 2 \\
        \hline
  \end{tabular}
  \caption{\label{tab:minitabelle}Eine Minitabelle}
 \end{center}
\end{table}

\section{Mathematische Formeln}

Mathematische Formeln können mit \LaTeX ~ ohne großen Aufwand
aufs Papier gebracht werden.

Hierzu genügen schon ein paar \LaTeX -Befehle:

\begin{center}
  \begin{math}
    \sqrt{\frac{\int_1^2 \vec a d\vec a}{\pi \sqrt{x}}}
  \end{math}
\end{center}

etwa entsteht durch \verb#\sqrt{\frac{\int_1^2 \vec a d\vec a}{\pi \sqrt{x}}}#.
Keine Angst. Das schaut zwar aus wie chinesisch, ist aber ganz einfach,
wenn man sich ein bischen eingearbeitet hat. Es gibt auch noch eine
Kurzschreibweise für den Mathe-Modus, diese schließt die Formeln in
\$-Zeichen ein.

\section{Das Literaturverzeichnis}

Mit Hilfe des Zusatztools {\tt BiBTeX} ist die Verwaltung von
Literatur-Verweisen recht einfach.  Hier ein kleines Beispiel, welches
einen Verweis auf \cite{kopkaEinf:92} verwendet. Einfach mal in {\tt
literatur.bib} schauen.

\section{Indizes}

Um ein Stichwortregister zu erstellen gibt es die Kommandos
\index{index} \verb#\index{}# und \verb#\makeindex#.  \index{makeindex}
\verb#\makeindex# kommt an den Anfang des \LaTeX -Dokuments und bewirkt,
daß die \verb#\index{}# Befehle nicht ignoriert, sondern zur Erstellung
einer Index-Datei verwendet werden.

Diese Indexdatei kann nun mit {\tt makeindex} weiterverarbeitet werden.
{\tt makeindex} ist ein Tool, das die von \LaTeX ~ erzeugte *.idx Datei
zu einer richtigen Index-Datei weiterverarbeitet, so daß die dann mit
\LaTeX ~ wieder gesetzt werden kann.

Zum Schluß möchte ich noch viel Erfolg bei der Diplomarbeit wünschen!


% --------------------------------------------------------------------
% Anhang
% --------------------------------------------------------------------
\clearpage
\begin{appendix}

%\include{appendix}

\chapter{Quelltexte}
Quelltexte im Anhang \dots
\ref{fig:demo}

%\chapter{Schaltpläne}
%
%Diese könnte man auch nur beilegen.
\end{appendix}

% --------------------------------------------------------------------
% Index
% --------------------------------------------------------------------
\clearpage
\printindex

% --------------------------------------------------------------------
% Literaturverzeichnis
% --------------------------------------------------------------------
% Nicht direkt referenzierte, aber benutzte Literatur
%\nocite{wasauchimmer}
%
\clearpage
%\bibliographystyle{geralpha}
\bibliographystyle{gerplain}
\bibliography{literatur} % welche bib-Dateien

% --------------------------------------------------------------------
% Dokument-Ende
% --------------------------------------------------------------------
\end{document}
%%%%%%%%%%%%%%%%%%%%%%%%%%%%%%
