\chapter{Introduction}\label{introduction}

Mobile indoor robots have become a common sight. Over 15 million units
of iRobot's Roomba vacuum robots have been sold since their introduction
in 2002
\footnote{http://www.irobot.com/About-iRobot/Company-Information/History.aspx}.
Applications are not limited to vacuuming: There are service robots,
entertainment robots, mapping robots, security robots, and many others.
Mobile indoor robots in commercial spaces are upheld as ``the next big
thing''
\footnote{http://spectrum.ieee.org/automaton/robotics/robotics-hardware/indoor-robots-for-commercial-spaces}.

Robotic navigation has also come a long way. The first vacuum robots
were merely equipped with a bumper sensor that let them to traverse a
room like a TV's DVD logo animation screensaver. In 2010, Neato Robotics
managed to develop and integrate a low-cost Lidar
\footnote{https://www.sparkfun.com/news/490} that first enabled path
planning for their vacuum robot. Some robots like LG's
Hom-bot\footnote{https://www.youtube.com/watch?v=UANWyiDf3hA} later used
ceiling cameras with visual slam (simultaneous localization and mapping)
for optical odometry information, but it was not before 2014 that iRobot
brought a vacuum robot to the market that used visual slam to create a
map of its environment
\footnote{https://www.technologyreview.com/s/541326/the-roomba-now-sees-and-maps-a-home/}.

There are still stories about vacuum robots failing hilariously in their
navigation capabilities, including one woman's hair being eaten as she
slept\footnote{https://www.theguardian.com/world/2015/feb/09/south-korean-womans-hair-eaten-by-robot-vacuum-cleaner-as-she-slept},
and of course countless examples of the joke ``What is worse than
finding dog poop on the carpet? Your robot vacuum finding dog poop on
the
carpet''\footnote{http://www.boredpanda.com/robot-vacuum-cleaner-spreads-dog-shit-everywhere/}.
Other obstacles that are really difficult to see with the sensors of
today's vacuum robots are very thin objects (like thin chair legs) in
the case of visual navigation, and transparent surfaces like windows and
glass walls with both visual and laser sensors. While theoretically it
would be possible to detect mirrors just the way humans do it, by
observing the geometry visible during movement and inferring the mirrors
position and orientation, no home robot has demonstrated that capability
so far.

Of course, the home robot owners want their robot to have better
navigational skills. A robot is more efficient if it can plan a path
with knowledge of all obstacles on it. That translates to quicker
cleaning. On the other hand it is safer, because the robot won't break
things and won't get hurt if it does not bump into potentially dangerous
things.

Thanks to recent efforts in miniaturization and falling costs, there is
a new option in the search for better navigation and obstacle sensors.
Usually associated with ship and aircraft localization, radar technology
made its way to the automotive industry in the form of weather-proof
range sensors. The newest sensors are even quite small (some sensors
come with antennas integrated in silicon) and cheap, with prices
projected to fall below 1€\cite{Brouwer2015}. Today's embedded
processors are also powerful enough to handle any kind of radar signal
processing. Combined with the easy licensing in the Industrial,
Scientific, Medical (ISM) frequency bands, it seems very worthwhile to
investigate how the use of radar will improve obstacle sensing in mobile
home robots.

To understand if, and how radar should be integrated in next-generation
home robots the next Section will give a brief overview of conventional
sensing techniques, followed by the theoretical background of radar
sensing. In section \#REF, reprojection mapping, a novel technique for
mapping with unsteered sensors, is proposed. Section \#REF then details
a proof-of-concept implementation of the method, Results of experiments
with this method are listed in Section \#REF and discussed and compared
in Section \#REF. Section \#REF finally concludes the thesis with an
outlook and how everything fits in the bigger picture of obstacle
sensing.