\chapter{Results and Evaluation}\label{results-and-evaluation}

\section{Evaluation}\label{evaluation}

\subsection{Evaluation dimensions}\label{evaluation-dimensions}

Needs to be * timely * false alarm rate * missed target rate * spatially
precise * see different types of obstacles * comparable to other sensors
* useful (humidity of wall is not interesting for obstacle detection)

\subsection{Evaluation scan targets}\label{evaluation-scan-targets}

\begin{itemize}
\tightlist
\item
  Cans (easy)
\item
  Walls
\item
  Glass walls
\item
  Chair legs
\item
  Cables on floor
\item
  Cliffs
\end{itemize}

\section{Results}\label{results}

During development of the reprojection method, over 30 scans were taken.
The environment for the scans is the BSH office in the Bosch Research
and Technology Center in Palo Alto. The office is fairly representative
of a typical office environment. It has carpet floors, desks, office
chairs, walls, corridors, and even glass walls.

The following list of scans is ordered by code name. It shows raw range
scans (down range echo intensity vs.~cross-range/mileage) for each scan,
parameters of the scan (orientation, sweep time), and resulting map. For
some of the scans, Lidar slam maps were recorded. They are overlaid in
the background of the reprojection map.

\ldots{}