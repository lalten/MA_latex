\chapter{Usage}\label{usage}

The following list shows the procedure for acquiring a radar reprojection map with the setup and implementation described in \cref{implementation-platform} and \cref{implementation}, respectively.

\begin{enumerate}
\def\labelenumi{\arabic{enumi}.}
\tightlist
\item
  Start robot, connect three sessions with \texttt{ssh\ zero2-pa}
\item
  Start ROS core with \texttt{roscore}
\item
  Start the node, with keyboard teleoperation and Cartographer Laser
  Slam:
  \texttt{roslaunch\ omniradar\ omniradar\_teleop\_lidarslam.launch}
\item
  Use \texttt{rqt} with the \emph{Dynamic Reconfigure} plugin to set the
  Omniradar node to generate the preferred number of sweeps and sweep
  duration. The configuration string can also be changed. The defaults
  are fine (One \SI{5}{ms} sweep)
\item
  \texttt{rosbag\ record\ /omniradar\_node/radar\_raw\ /odom\ /tf\ /map\ -O\ scan}
  will record a rosbag ``scan.bag'' containing all ROS messages with
  radar data, Kobuki odometry and slam map and transforms from
  Cartographer.
\item
  In the terminal running \texttt{roslaunch}, use the arrow keys to move
  the robot around (\(\uparrow\) and \(\downarrow\) to increase and
  decrease speed, \(\leftarrow\) and \(\rightarrow\) to increase and
  decrease rotation speed). \(E\) resets speed to zero and makes the
  robot halt.
\item
  After recording some interesting data, stop the rosbag record
  (\(Ctrl+C\)). Open a new terminal on your local machine and run
  \texttt{ssh\ zero2-pa\ "tar\ zcf\ -\ scan.bag"\ \textbar{}\ tar\ zxf\ -}.
  The rosbag will be transferred to your machine. Using ssh with the tar
  pipe is the fastest way to transfer the data (around \SI{100}{Mbps} on
  the BSH wifi). Compressing first (\texttt{rosbag\ compress\ scan.bag})
  and then sending takes a while on the not-so-powerful Odroid platform.
\item
  Optionally filter out unwanted transforms from the rosbag to speed up
  later processing:
  \texttt{rosbag\ filter\ scan.bag\ scan\_filtered.bag\ \textquotesingle{}(topic\ ==\ "/tf"\ and\ m.transforms{[}0{]}.child\_frame\_id\ ==\ "odom")\ or\ topic\ !=\ "/tf"\textquotesingle{}}
\item
  In Matlab, run
  \texttt{radar\_data\ =\ radar\_bag2array("/path/to/your/scan.bag");}.
  This will read the bag sequentially into memory and extract the data:
  The robot position is recorded from odometry information and corrected
  using the /map \(\rightarrow\) /odom transform as reported by slam
  localization. Cross range mileage is calculated as cumulative sum of
  distance between radar positions (as the radar is not mounted over the
  robot's rotation center, the radar mileage is different from robot
  mileage as soon as any rotational velocity is present). Lastly, all
  values are interpolated at the radar message timestamps. It is a good
  idea to save the function's output, using
  \texttt{save(\textquotesingle{}radar\_data/radar\_data\_scan.mat\textquotesingle{},\ \textquotesingle{}radar\_data\textquotesingle{})}
\item
  The radar data can now be analyzed. The
  \texttt{plot\_all} script can be used to get a good
  overview over raw data, Doppler speeds, direction of arrival, and
  reprojection map. If a lidar slam occupancy gridmap is available, the
  reprojection map will also be overlaid on it.
\end{enumerate}
