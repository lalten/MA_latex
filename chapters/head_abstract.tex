\begin{abstract}
    Mobile indoor robots mostly rely on lidar and vision sensors to remotely detect obstacles in their path. These sensors have trouble detecting some common real-world obstacles like transparent surfaces and chair legs. Recent affordable near-range miniature radar sensors enable new solutions.\\
    This thesis introduces a simple solution for a radar sensor being moving through a static environment without the need for beamforming or a mechanically scanning radar. The idea is put to test in experiments with an Omniradar FMCW radar mounted on a Kobuki robot platform. A comparison with lidar and RGBD sensors shows how promising the setup is for mapping unstructured static environments.
    
    ~\\
    
    Mobile Indoor-Roboter nutzen vorallem Lidar und Vision Sensoren um Hindernisse auf dem Weg aus der Entfernung zu erkennen. Diese Sensoren können einige für die echte Welt relevante Hindernisse wie transparente Flächen und Stuhlbeine jedoch nicht detektieren. Neu verfügbare Miniatur-Radarsensoren für Kurzreichweiten ermöglichen hier neue Lösungen.\\
    Diese Arbeit präsentiert eine unkomplizierte Lösung zur Kartierung statischer Umgebungen mit einem Radarsensor. Der Sensor muss bewegt, aber nicht gescannt werden und macht so Beamforming und mechanische Drehung überflüssig. Das Konzept wird in Experimenten mit einem Omniradar FMCW Radarsensor auf einer Kobuki Roboterplattform validiert. Der Vergleich mit Lidar und RGBD-Sensoren zeigt das Potential des Setups für die Erkundung unstrukturierter Umgebungen.
\end{abstract}