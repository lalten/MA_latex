\usepackage{bibgerm}		% Für geralpha.bst. MUSS vor babel stehen
\usepackage[utf8]{inputenc}	% Latin1 input encoding (umlauts)
\usepackage[german,english]{babel} % Mehrsprachiger Text, Reihenfolge wichtig!
				   % Die zuletzt genannte Sprache wird verwendet!
\usepackage{version}		% handle comments and different versions
				% \begin/end{comment},
				% \include/excludeversion{foo}
\usepackage{parskip}		% seperate paragraphs by skip (blank line)
%\usepackage{amsfonts,amsmath,amssymb,array} % choose if needed
%\usepackage{ae,aecompl}	% Benutze 'Almost European' Zeichensatz         
%\usepackage{type1cm}	% Computer Modern type 1 fonts
\usepackage{lmodern}	% Latin Modern nehmen, da mehr Glyphen vorhanden 
\usepackage[T1]{fontenc}	% Verwende T1-Kodierung für Zeichensätze        

\usepackage{textcomp} % to make ° work


\usepackage{changebar}
\usepackage{makeidx}\makeindex

\usepackage{graphicx}


% https://tex.stackexchange.com/questions/42619/x-mark-to-match-checkmark
\usepackage{pifont}% http://ctan.org/pkg/pifont
\newcommand{\cmark}{\ding{51}}%
\newcommand{\xmark}{\ding{55}}%

\usepackage{longtable}
\usepackage[table]{xcolor}
\definecolor{alternatedrowcolor}{rgb}{0.93,0.95,1.0}%{gray}{0.9}
\usepackage{booktabs}
\usepackage{tabu}

\usepackage{pdflscape} % for 'landscape' environment

\usepackage{subcaption}

\usepackage{amsmath}